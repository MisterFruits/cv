\documentclass[12pt]{lettre}

\usepackage[utf8]{inputenc}
\usepackage[T1]{fontenc}
%\usepackage{lmodern}
\usepackage[frenchb]{babel}


\usepackage{eurosym}
%%%%%%%%%%%%%%%%%%%%%%%%%%%%%%%
%%% LE FORMAT DE LA PAGE
%%%%%%%%%%%%%%%%%%%%%%%%%%%%%%%
%\geometry
%   {%
%   paper=a4paper,%
%   margin=3cm,%
%   hoffset=0cm,%
   %headsep=0.5cm,%
%   }%
%%%%%%%%%%%%%%%%%%%%%%%%%%%%%%%
%%% EN-TÊTES
%%%%%%%%%%%%%%%%%%%%%%%%%%%%%%%
%\pagestyle{empty}
%%%%%%%%%%%%%%%%%%%%%%%%%%%%%%%
%%% OPTIONS DES TABLEAUX
%%%%%%%%%%%%%%%%%%%%%%%%%%%%%%%
%\setlength{\tabcolsep}{1pt}
%%%%%%%%%%%%%%%%%%%%%%%%%%%%%%%
%%% OPTIONS DIVERSES : math, soulignement
%%%%%%%%%%%%%%%%%%%%%%%%%%%%%%%
\parindent=6ex

%\usepackage[margin={0.6in,0.5in}]{geometry}
% \addtolength{\textheight}{4cm}
%\AtBeginDocument{\recomputelengths}


\begin{document}

\begin{letter}{A l'attention de\\Jean-Michel Dussauze,\\Bertrand Cabon et\\ Loic Boussouf}
\name{Victor Cameo Ponz}
\address{Victor Cameo Ponz\\47 rue Guynemer\\31200 Toulouse}
\email{victor.cameo\_ponz@insa-rouen.fr}
\telephone{+33 (0)6.74.08.96.90}
\lieu{Toulouse}
\nofax

\conc{Candidature au poste d'ingénieur developpement logiciel et recherche opérationnelle}
\opening{Messieurs}
%\setlength{\parindent}{\openingindent}%corps
J'ai pris connaissance d'une offre pour un poste d'ingénieur developpement logiciel et recherche opérationnelle dans le département TSOEC34.
Sa description a retenue toute mon attention car elle a attrait à beaucoup de domaines qui me tiennent à c\oe{}ur.
Passionné par l'algorithmie et l'optimisation, j'ai effectué mon stage chez Astrium entreprise a sein de laquelle je me suis éveillé au monde du spacial.
% auquel j'ai envie d'apporter ma contribution.
% J'ai pu collaborer avec des équipes compétentes et motivées, et de développer des premières solutions dont les résultats sont très encourageants.
% pas ici -> candidature pas ouverte = aurevoir
% Même si il n'est actuellement ouvert à des candidature externe cette offre m'intéresse beacoup.

Durant les 6 premiers mois passés à astirum j'ai travaillé en colaboration
avec Vincent Tugend (TELECOM), Bertrand Cabon et Fawzi Bessaih sur les acitivités Bebop.
Ces sujets sont au cœur de l'activité de proposition d'Astrium,
et beaucoup de problématiques restent à restent à explorer.
Par exemple la prise en compte des allocations de porteuses dans les
dimensionnements système ou encore l'optimisation de demande de capacités
Il me semble que ces axes sont aujourd'hui stratégiques pour Airbus Defence and Space.

Mon récent travail sur Pacoma m'a permis d'apréhender l'aspect production informatique. Pacoma bla bla.
travail sur pacoma =  mise en production d'un logiciel + méthode de developpemnts efficace ->
projet avec nico: dans modélisation d'une payload de satellite en java => intéret pour la réutilisation dans les outils du département
génération de procédures opérationnelles

plus d'implication dans les project, véhiculer mes idée de manière plus efficace. S'engager dans le spacial
les sujets d'optimisation comme le beamlayout ou encore l'allocation frequencielles pour satellites multibeams sont des sujets qui me passionne.

Mon stage technitien = caractérisation de signaux electrique grace à des réseaux de neuronnes artificiels. -> très proches de certaines prblèmatiques traitement du signal, test de matériel AIT (stage clem...) en rapport avec l'ESA et le CNES


intro : Travaille chez astrium depuis 1 an, stage en optimisation de 6 mois
Cela m'a permis de découvrir un environnement multi-disciplinaire très riche, dans lequel j'évolue depuis presque un an et demi.


Le stage que j'ai effectué à Astrium au cours de ces six derniers mois a conforté ma volonté de travailler dans le domaine des mathématiques appliquées et de l'optimisation. Son exécution m'a permis de découvrir les problématiques complexes de l'industrie spatiale qui m'étaient inconnues. J'ai eu la chance d'intégrer l'équipe de management et d'optimisation de la charge utile des satellites de télécommunication. J'y ai développé en autonomie un algorithme d'optimisation ainsi que des outils de visualisation et d'analyse de résultats.

La formation que j'ai suivie à l'INSA m'a donné de fortes compétences en modélisation mathématique, abstraction et de généralisation. Je suis dédié à travaillé dans des domaines transverses grâce à ma capacité à transposer une méthode d'une discipline à une autre. Les projets et stages que j'ai effectué au cours de ma formation en sont l'illustration parfaite. De plus, les compétences en informatique et gestion de projet que j'ai développé me permettent d'être à l'aise et efficace dans tous types d'environnements techniques.% Elle sont un véritable atout dans les applications que j'ai pu de faire de l'enseignement qu'on m'a donné.

Ma première expérience dans le monde du spatiale a été une expérience enrichissante et motivant et l'opportunité d'intégrer une entreprise  telle que Thales me permettrait de m'y investir complètement.
%
Je me tiens à votre disposition pour toutes informations complémentaire ou un entretient.
\closing{Je vous prie d'agréer, Madame, Monsieur, l'expression de mes sentiments distingués.}
%\encl{Curiculum Vitae}
\end{letter}

\end{document}
