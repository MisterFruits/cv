\documentclass[12pt]{lettre}

\usepackage[utf8]{inputenc}
\usepackage[T1]{fontenc}
%\usepackage{lmodern}
\usepackage[frenchb]{babel}


\usepackage{eurosym}
%%%%%%%%%%%%%%%%%%%%%%%%%%%%%%%
%%% LE FORMAT DE LA PAGE
%%%%%%%%%%%%%%%%%%%%%%%%%%%%%%%
%\geometry
%   {%
%   paper=a4paper,%
%   margin=3cm,%
%   hoffset=0cm,%
   %headsep=0.5cm,%
%   }%
%%%%%%%%%%%%%%%%%%%%%%%%%%%%%%%
%%% EN-TÊTES
%%%%%%%%%%%%%%%%%%%%%%%%%%%%%%%
%\pagestyle{empty}
%%%%%%%%%%%%%%%%%%%%%%%%%%%%%%%
%%% OPTIONS DES TABLEAUX
%%%%%%%%%%%%%%%%%%%%%%%%%%%%%%%
%\setlength{\tabcolsep}{1pt}
%%%%%%%%%%%%%%%%%%%%%%%%%%%%%%%
%%% OPTIONS DIVERSES : math, soulignement
%%%%%%%%%%%%%%%%%%%%%%%%%%%%%%%
\parindent=6ex

%\usepackage[margin={0.6in,0.5in}]{geometry}
% \addtolength{\textheight}{4cm}
%\AtBeginDocument{\recomputelengths}


\begin{document}

\begin{letter}{A l'attention de\\Jean-Michel Dussauze\\et Bertrand Cabon}
\name{Victor Cameo Ponz}
\address{Victor Cameo Ponz\\16 rue Payras\\31000 Toulouse}
\email{victor.cameo\_ponz@insa-rouen.fr}
\telephone{+33 (0)6.74.08.96.90}
\lieu{Toulouse}
\nofax

\conc{Candidature au poste d'ingénieur recherche opérationnelle et developpement logiciel.}
\opening{Messieurs}
%\setlength{\parindent}{\openingindent}%corps
{% Intro
J'ai pris connaissance de l'offre d'ingénieur recherche opérationnelle et
developpement logiciel, ref : 10260005 AD FR EXT 1.
Sa description a retenue toute mon attention car elle englobe beaucoup de
domaines qui me tiennent à c\oe{}ur.
Passionné par l'algorithmie et l'optimisation, j'ai effectué mon stage
chez Astrium entreprise a sein de laquelle je me suis éveillé au monde du spacial.
}

{% Bebop
Durant ces 6 premiers mois passés à Astirum j'ai travaillé en colaboration
avec Vincent Tugend, Bertrand Cabon et Fawzi Bessaih sur les acitivités
d'optimisation d'allocation fréquentielle pour satellite multibeams.
Ce sujets est au cœur de l'activité de proposition d'Astrium,
et beaucoup de problématiques restent à restent à explorer.
Par exemple la prise en compte des allocations de porteuses dans les
dimensionnements système ou encore l'optimisation de demande de capacités.
Il me semble que ces axes sont aujourd'hui stratégiques pour Airbus Defence and Space.
}

{% Pacoma
Je suis actuellement en charge du developpement de Pacoma au sein
du département TSOEC34.
C'est un logiciel opérationnel de reconfiguration de la charge utile de
télécomunication.
Ce poste m'a permis d'apréhender le processus complet d'industrialisation d'un
logiciel informatique.
Cotoyé par des équipes compétentes et motivées, nous travaillons à la
mutualisation des outils de modélisation et d'assistance à la conception de
satellites.
}

{% INSA & spacial
blabla sur mon intérêt pour le monde complexe et technique du spacial qui fit bien avec le caractère ouvert de ma formation à l'insa. Diversité technique au sein d'ADS
La formation que j'ai suivie à l'INSA m'a donné de fortes compétences en
modélisation mathématique, abstraction et de généralisation.
Je suis dédié à travaillé dans des domaines transverses grâce à ma
capacité à transposer une méthode d'une discipline à une autre.
}

{% Implication inovation
Obtenir ce poste chez Airbus Defence and Space donnerai un sens plus fort
à mes activité pro et accroplus d'implication dans les project, véhiculer mes idée de manière plus efficace. S'engager dans le spacial
faire valoir els compétence que je n'ai pas encore pu exploiter
Mon stage technitien = caractérisation de signaux electrique grace à des réseaux
de neuronnes artificiels. -> peuvent se rapprocher de certaines prblèmatiques
traitement du signal, test de matériel AIT (stage/these clem...) en rapport avec
l'ESA et le CNES. LVQ \& backpropagation
}

hate de vous rencontrer pour parler avec vous des détails exacts des actions
requise par ce poste

\closing{Je vous prie d'agréer, Messieurs, l'expression de mes sentiments distingués.}
%\encl{Curiculum Vitae}
\end{letter}

\end{document}
