\documentclass[12pt]{lettre}

\usepackage[utf8]{inputenc}
\usepackage[T1]{fontenc}
%\usepackage{lmodern}
\usepackage[frenchb]{babel}

\usepackage{nopageno} % suppress page numbers
\usepackage{eurosym}
%%%%%%%%%%%%%%%%%%%%%%%%%%%%%%%
%%% LE FORMAT DE LA PAGE
%%%%%%%%%%%%%%%%%%%%%%%%%%%%%%%
%\geometry
%   {%
%   paper=a4paper,%
%   margin=3cm,%
%   hoffset=0cm,%
   %headsep=0.5cm,%
%   }%
%%%%%%%%%%%%%%%%%%%%%%%%%%%%%%%
%%% EN-TÊTES
%%%%%%%%%%%%%%%%%%%%%%%%%%%%%%%
%\pagestyle{empty}
%%%%%%%%%%%%%%%%%%%%%%%%%%%%%%%
%%% OPTIONS DES TABLEAUX
%%%%%%%%%%%%%%%%%%%%%%%%%%%%%%%
%\setlength{\tabcolsep}{1pt}
%%%%%%%%%%%%%%%%%%%%%%%%%%%%%%%
%%% OPTIONS DIVERSES : math, soulignement
%%%%%%%%%%%%%%%%%%%%%%%%%%%%%%%
\parindent=6ex

%\usepackage[margin={0.6in,0.5in}]{geometry}
% \addtolength{\textheight}{4cm}
%\AtBeginDocument{\recomputelengths}


\begin{document}

\begin{letter}{A l'attention de M. Francis Daumas\\Directeur du CINES\\950 rue de Saint Priest\\34 000 Montpellier}
\name{Victor Cameo Ponz}
\address{Victor Cameo Ponz\\16 rue Peyras\\31000 Toulouse}
\email{victor.cameo\_ponz@insa-rouen.fr}
\telephone{+33 (0)6.74.08.96.90}
\lieu{Toulouse}
\nofax

\conc{Candidature au poste d'ingénieur de recherche, support HPC et projet européen -- ref, DCIIGRSUPPRA.}
\opening{Monsieur le directeur,}
%\setlength{\parindent}{\openingindent}%corps
2 points dans la lettre = motivation et compétence à apporté (expérience)

argument:
capacité à communiquer facilement, en français, en anglais.

intéret pour les projets d'orizon différent. Curieux.
grande plasticité et facilité d'intégration à des projets d'orizons différents
capacité d'adaptation

Intéret pour la recherche, l'enseignement = un objectif (pas dans le court terme) ?? envie de participer à la connaissance mondiale: partager le savoir (publication ?) renforcer mes propres connaissances afin de pouvoir les transmettres

disponible et mobile en france ou à l'étranger, intéret pour les séminaire, choses nouvelles. Aime beaucpou apprendre

% suivi avec intéret le developpement de certaine application parallèle chez Airbus defence & space (plus un truc à dire qu'à mettre)
% ainsi que plusieurs cours à propos de l'algorithmie poussée à l'INSA (à l'oral)
je comprends les principes de la programmation parallèle et ai déjà aqui les bases avec certaines librairie de calcul comme OpenMP et MPI. + Bases théoriques = lois reliant vitesse d'execution du code avec le nombre de coeur et l'accès aux données (problèmatique de cache...)

passioné d'informatique et de mathématique. interet pour la veille techno

aucun problème pour travailler dans des environements déporté (sur server...) bonne utilisation des outils UNIX

expérience dans l'industrialisation d'outils d'optimisation informatique. Pacerelles C/C++ - Java ou Python.

Conscient du problème posé par le traitement des données en masse (cloud computing...)

fait contrets qui prouve le savoir faire:
- relation client entre les différente boite de services et département à astrium
je suis dans un projet arbre (git ??)
- rapport de stage en anglais, expérience de 3 mois à l'internationnal
-mobilité = facteur de motivation,  relation internationnale

motivation
déplacement, projet européen, cdi to cdd

% discuter des perspectives à long terme. Ca m'intéresse mais je veux continuer dans le domaines, quid d'un renouvellement ? subvention européennes ? à demander en entretient


beaucoup de relationnel dans le poste = axes de la lettre. Comprendre le besoin des uns des autre. Savoir se mettre dans le context d'un projet.
travail en réseau, adaptation à un client (boite de sous traiance) j'aime la relation humaine.

\newpage


% Intro
{
Je connais les activités du département de calcul intensif du CINES pour en
avoir déjà discuté avec Gabriel Hautreux qui y travaille depuis un an
maintenant.
Aussi, il m'a parlé de l'offre de poste d'ingénieur de recherche,
support HPC et projet européen -- reféférencée, DCIIGRSUPPRA,
qui a retenue toute mon attention.
}

% Expérience
{
Sortant de l'INSA (Institut National des Science Appliquées) qui m'a donné
une formation en mathématiques et informatique scientifique, je travaille
actuellement chez AKKA pour le compte d'Airbus Defence and Space.
Depuis un an j'y maintient un produit opérationnel d'optimisation
basé sur un noyau de calcul codé en C++ et doté d'une interface graphique élaborée en Java.
Ce logiciel est livré à des clients externe d'Airbus, et suit donc un processus de qualité ou j'interviens à toutes les étapes.
% Depuis l'évolution des fonctionnalités de l'ensemble du logiciel

J'ai aussi pu, durant ma formation, réaliser un stage à l'Université Fédérale d'Uberlandia au Brésil. Mon role était alors de d'effectuer une comparaison de deux réseaux de neuronnes artificiels utilisé pour la simulation de prothèse en réalité augmenté. J'ai établit -- plan de test -- rapport technique destiné à la publication.
% compétence technique
 - 2 expériences professionnelle en C++.
 Chez Aievus DS je travaille sur un project avec un noyau en C++ pré-existant sur lequel j'ai du monter en compétence et y ajouter des fonctionnalité. Ce dev incluant test/Qualité et livraison à un client externe d'ads.
 Benchmarking de deux réseaux de neuronnes artificiels. Un en python et l'autre en C++. Ecriture d'un rapport technique destiné à la publication.

 - à l'aise avec les environnement UNIX. COnnaissance et pratique des environnements unix. Usage quotidient langue maternelle...

- gros background du à la dominante des mes étude en mathématique optimisation

% compétence communication travail en réseu, relationnel avec d'autre domaines d'expertise que le mien.
 - A l'aise à l'anglais. J'ai eu une expérience à l'internationnal au Brésil qui à aboutit à un un rapport de stage en Anglais. + rapport de stage en anglais astrium. je travaille en anglais

 Colaboeration médecine informatique.

 Expérience du travail avec des expert / chercheur en medecien et en satellites de télécomunication. Facilité à s'intégrer dans un environnement informatique complexe. J'ai éprouvé mes capacité à travailler en réseau et à comprendre les demandes des client. Enormément d'échange. Type d'organisation du projet en AGILITé
}

% Motivation
{
 faire des maths et d'opti. avec des outils reconnu et puissant. Je vais pouvoir faire des maths et de l'opti à un niveau poussé. aspect recherche et de pointe. Je suis intéressé pour bosser chez vous car il y a les moyens. Je vois le potentiel, eveil mon intéret quasi exitation. Perspec de travailler sur de tel outils dont dispose le centre est un elément très important (déclencheur) dans ma décision de me porter candidat.

 la dimension internationnal/mobilité du poste. Le relationnel. projet varié
 interet pour le travail de pointe.
}


\closing{Je vous prie d'agréer, Messieurs, l'expression de mes sentiments distingués.}
%\encl{Curiculum Vitae}
\end{letter}

\end{document}


je veux poser ma candidature; estce encore possible niveau date;
à qui s'adresser pour avoir plus d'infos
