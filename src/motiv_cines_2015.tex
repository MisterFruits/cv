\documentclass[10pt,a4paper]{lettre}

\usepackage[utf8]{inputenc}
\usepackage[T1]{fontenc}
%\usepackage{lmodern}
\usepackage[frenchb]{babel}

\usepackage{nopageno} % suppress page numbers
\usepackage{eurosym}
%%%%%%%%%%%%%%%%%%%%%%%%%%%%%%%
%%% LE FORMAT DE LA PAGE
%%%%%%%%%%%%%%%%%%%%%%%%%%%%%%%
%\geometry
%   {%
%   paper=a4paper,%
%   margin=3cm,%
%   hoffset=0cm,%
   %headsep=0.5cm,%
%   }%
%%%%%%%%%%%%%%%%%%%%%%%%%%%%%%%
%%% EN-TÊTES
%%%%%%%%%%%%%%%%%%%%%%%%%%%%%%%
%\pagestyle{empty}
%%%%%%%%%%%%%%%%%%%%%%%%%%%%%%%
%%% OPTIONS DES TABLEAUX
%%%%%%%%%%%%%%%%%%%%%%%%%%%%%%%
%\setlength{\tabcolsep}{1pt}
%%%%%%%%%%%%%%%%%%%%%%%%%%%%%%%
%%% OPTIONS DIVERSES : math, soulignement
%%%%%%%%%%%%%%%%%%%%%%%%%%%%%%%
\parindent=18ex

%\usepackage[margin={0.6in,0.5in}]{geometry}
% \addtolength{\textheight}{4cm}
%\AtBeginDocument{\recomputelengths}
\addtolength{\textwidth}{2.5cm}
\addtolength{\oddsidemargin}{-2cm}


\begin{document}

\begin{letter}{A l'attention de M. Francis Daumas\\Directeur du CINES\\950 rue de Saint Priest\\34 000 Montpellier}
\name{Victor Cameo Ponz}
\address{Victor Cameo Ponz\\16 rue Peyras\\31000 Toulouse}
\email{victor.cameo\_ponz@insa-rouen.fr}
\telephone{+33 (0)6.74.08.96.90}
\lieu{Toulouse}
\nofax

\conc{Candidature au poste d'ingénieur de recherche, support HPC et projet européen -- ref, DCIIGRSUPPRA.}
\opening{Monsieur le directeur,}

% Intro
{
Je connais les activités du département de calcul intensif du CINES pour en
avoir discuté avec Gabriel Hautreux qui y travaille depuis un an
maintenant.
Par son biais j'ai eu connaissance de l'ouverture d'un poste d'ingénieur de recherche,
support HPC et projet européen -- reféférencée DCIIGRSUPPRA,
qui a retenue toute mon attention.
Diplomé de l'INSA (Institut National des Science Appliquées) en
mathématiques appliquées informatique scientifique je m'intéresse
spécifiquement à l'algorithmie et à l'optimisation.

% Expérience
{
Je travaille
actuellement chez AKKA pour le compte d'Airbus Defence and Space.
Depuis un an j'y maintiens un produit opérationnel d'optimisation
basé sur un noyau de calcul codé en C++ et doté d'une interface graphique élaborée en Java.
Ce logiciel est livré à des clients externes d'Airbus. Il suit un
% un mix des trois prochaines ligne ?
processus de développement et de qualité où
% j'interviens à tous les niveaux.
je suis acteur depuis les spécifications avec le client jusqu'à la livraison.
% plus insister sur la partie c++ ? nop peut-etre pas
% Chez Aievus DS je travaille sur un project avec un noyau en C++ pré-existant sur lequel j'ai du monter en compétence et y ajouter des fonctionnalité
% plutot insister sur le fait que je maitrise l'info et le process de dev?


Durant ma formation, j'ai eu l'opportunité de réaliser un stage à
l'Université Fédérale d'Uberlandia au Brésil.
Mon rôle était alors de compararer de deux réseaux de neurones
artificiels utilisés pour la simulation d'une prothèse de membre
suppérieur.
J'ai mis en situation les codes Python et C dans le but de comprendre
leurs qualités et leurs défauts dans le cadre précis de la réalité
augmentée appliquée à la médecine. % publication ? non après avec l'anglais

Mes différentes expériences m'ont permis de comprendre les besoins des
utilisateurs de codes de calcul que j'ai pu rencontrer. Ma capacité
d'adaption à ces domaines scientifiques variés et ma capacité
d'écoute et de dialogue ont été les clés de la réussite de ces projets.
De plus j'utilise quotidiennement des outils de développement et je
travail en équipe dans un environnement internationnal,
prêt à faire part de mon dynamisme et à partager mes connaissances.
}

% Motivation
{
La perspective de travailler au département de calcul intensif du CINES
avec des outils à la pointe de la technologie comme Occigen correspond à mes attentes professionnelles.
C'est un des points, avec la dimension européenne de la mission qui
motive ma candidature pour ce poste.
J'espère vous avoir communiqué mon enthousiasme et je souhaitrais vous
rencontrer pour un entretient.
}

\closing{Je vous prie d'agréer, monsieur, l'expression de mes sentiments distingués.}
%\encl{Curiculum Vitae}
\end{letter}

\end{document}
evoluer dans un milieu de recherche ou mes compétence en math info pourront mieux s'exprimer que dans milieu industriel

remplacer developpeur java dans le CV
%%%%%%%%%%%%%%%%%%%%
%% NOTES          %%
%%%%%%%%%%%%%%%%%%%%

J'ai éprouvé mes capacités à travailler dans des domaines transverse et à
comprendre les ***demandes/problèmes*** d'expert de tous les horizons.
Ainsi j'ai pu participer à plusieurs projects mettant en place une méthode agile. % comment faire glu entre ces phrases ?
Mon expérience à l'international et le fait que je travaille en anglais
m'ont permis d'être à l'aise dans cette langue à l'écrit comme à l'oral.
% compétence communication travail en réseu, relationnel avec d'autre domaines d'expertise que le mien.
Expérience du travail avec des expert / chercheur en medecien et en satellites de télécomunication. Facilité à s'intégrer dans un environnement informatique complexe. J'ai éprouvé mes capacité à travailler en réseau et à comprendre les demandes des client. Enormément d'échange. Type d'organisation du projet en AGILITé
tout en pofinant mes compétences technique:

\newpage
% compétence technique
 - 2 expériences professionnelle en C++.
 Chez Aievus DS je travaille sur un project avec un noyau en C++ pré-existant sur lequel j'ai du monter en compétence et y ajouter des fonctionnalité. Ce dev incluant test/Qualité et livraison à un client externe d'ads.
 Benchmarking de deux réseaux de neuronnes artificiels. Un en python et l'autre en C++. Ecriture d'un rapport technique destiné à la publication.


%\setlength{\parindent}{\openingindent}%corps
2 points dans la lettre = motivation et compétence à apporté (expérience)

argument:
capacité à communiquer facilement, en français, en anglais.

intéret pour les projets d'orizon différent. Curieux.
grande plasticité et facilité d'intégration à des projets d'orizons différents
capacité d'adaptation

Intéret pour la recherche, l'enseignement = un objectif (pas dans le court terme) ?? envie de participer à la connaissance mondiale: partager le savoir (publication ?) renforcer mes propres connaissances afin de pouvoir les transmettres

disponible et mobile en france ou à l'étranger, intéret pour les séminaire, choses nouvelles. Aime beaucpou apprendre

% suivi avec intéret le developpement de certaine application parallèle chez Airbus defence & space (plus un truc à dire qu'à mettre)
% ainsi que plusieurs cours à propos de l'algorithmie poussée à l'INSA (à l'oral)
je comprends les principes de la programmation parallèle et ai déjà aqui les bases avec certaines librairie de calcul comme OpenMP et MPI. + Bases théoriques = lois reliant vitesse d'execution du code avec le nombre de coeur et l'accès aux données (problèmatique de cache...)

passioné d'informatique et de mathématique. interet pour la veille techno

aucun problème pour travailler dans des environements déporté (sur server...) bonne utilisation des outils UNIX

expérience dans l'industrialisation d'outils d'optimisation informatique. Pacerelles C/C++ - Java ou Python.

Conscient du problème posé par le traitement des données en masse (cloud computing...)

fait contrets qui prouve le savoir faire:
- relation client entre les différente boite de services et département à astrium
je suis dans un projet arbre (git ??)
- rapport de stage en anglais, expérience de 3 mois à l'internationnal
-mobilité = facteur de motivation,  relation internationnale

motivation
déplacement, projet européen, cdi to cdd

% discuter des perspectives à long terme. Ca m'intéresse mais je veux continuer dans le domaines, quid d'un renouvellement ? subvention européennes ? à demander en entretient


beaucoup de relationnel dans le poste = axes de la lettre. Comprendre le besoin des uns des autre. Savoir se mettre dans le context d'un projet.
travail en réseau, adaptation à un client (boite de sous traiance) j'aime la relation humaine.

\newpage




je veux poser ma candidature; estce encore possible niveau date;
à qui s'adresser pour avoir plus d'infos

l'interet du python en language scientifique: scripting/interface facile avec le C
