\documentclass[10pt,a4paper,sans]{moderncv}%sans(serif)/roman
\moderncvtheme[blue]{classic} %blue orange red gray green
\usepackage[utf8]{inputenc}
\usepackage[scale=0.85]{geometry}
%\usepackage[margin={0.6in,0.5in}]{geometry} %si limite en place
\addtolength{\textheight}{4cm}
\AtBeginDocument{\recomputelengths}


%----------------------------------------------------------------------------------
%            Commandes perso
%----------------------------------------------------------------------------------
\renewcommand*{\cvlanguage}[4]{%
  \cvdoubleitem{\textbf{#1}}{\small#2}{\textbf{#3}}{\small#4}}

% usage: \cventry{years}{degree/job title}{institution/employer}{localization}{optionnal: grade/...}{optional: comment/job description}
\renewcommand*{\cventry}[6]{%
  \cvline{#1}{%
    {\bfseries#2}%
    \ifx#3\else{, {#3}}\fi%
    \ifx#4\else{, \small#4}\fi%
    \ifx#5\else{, \small#5}\fi%
    .%
    \ifx#6\else{\newline{}\begin{minipage}[t]{\linewidth}\small#6\end{minipage}}\fi
    }}%

%----------------------------------------------------------------------------------
%            Photo et infos persos
%----------------------------------------------------------------------------------
\firstname{Victor}
\familyname{Cameo Ponz}
\title{Ingénieur INSA, mathématiques appliquées}
\address{53 rue Georges Guynemer}{31 200 Toulouse}
\mobile{+33 6 74 08 96 90}
\email{vicameo@free.fr}
\extrainfo{Titulaire du permis B, 30 ans}
%\photo[60pt]{./photo/victorID.JPG}
%\quote{Ingénieur d'étude en optimisation et recherche opérationnelle, disponible immédiatement.}

%----------------------------------------------------------------------------------
%            contenu du cv
%----------------------------------------------------------------------------------
\begin{document}
\maketitle
\vspace{-1cm} %si limite en place

\section{Expérience professionnelle}
\cventry{2015 - 2018}
    {Ingénieur de recherche en calcul haute performance (HPC)}
    {CINES}
    {Montpellier (34)}
    {3 ans}
    {
\textbf{Support, maintenance et tests de performances} sur les machines du Centre Informatique National de l'Enseignement Supérieur (CINES). \textbf{Encadrement d'une équipe de 10 personnes} pour l'évaluations de performances sur bancs d'essais sur \textbf{Xeon Phi, GPU, CPU}.
\textbf{Formateur} optimisation et parallélisation de code et initiation à l'outil CVS git.
% Encadrement du groupe de travail européen pour l'évaluations de performances sur bancs d'essais sur \textbf{Xeon Phi, GPU, CPU} (PRACE UEABS, 10 personnes). Support, maintenance, modernisation sur les machines du CINES.
% Définition d'une suite d'une suite de benchmark européenne pour accélérateurs.
% Support, maintenance, modernisation et tests de performance sur les machines du CINES.
% Porteur du code Specfem3D dans le cadre de la cellule de veille technologique de GENCI et des suites de benchmark PRACE.
% Aide au \textbf{portage et à l'optimisation de code} pour des SME (PRACE SHAPE).
    }
\cventry{2013 -- 2015}
    {Ingénieur optimisation et produits d'opérations satellites}
    {AKKA pour Airbus}
    {Toulouse (31)}
    {2 ans}
    {
        Développement et production d'un logiciel de \textbf{reconfiguration de la charge utile} d'un satellite de télécomunication. Génération de procédures opérationnelles et \textbf{intégration} sur centre de contrôle. \textbf{JAVA et C++}.
        % \begin{itemize}
            % \item Noyau d'optimisation en C++
        %     \item Génération de procédures opérationnelles
        %     \item Intégration sur centre de contrôle
        % \end{itemize}
    }
\cventry{2013}
    {Stage optimisation et produit de conception de satellites}
    {EADS Astrium}
    {Toulouse (31)}
    {6 mois}
    {
        Optimisation de \textbf{l'allocation fréquentielle} d'un \textbf{satellite multi-beam}.
        \begin{itemize}
            \item Retro-engineering et amélioration d'un noyau codé en programmation par contraintes.
            \item \textbf{Conception et réalisation d'un solveur} recuit simulé.
            \item Développement d'outils de visualisation avec JFreeChart et World Wind Java.
        \end{itemize}
    }
% \cventry{2011 -- 2012}
%     {Professeur particulier}{}
%     {Rouen (76)}{}
%     {Cours particuliers de mathématiques.}
\cventry{2011}
    {Stage mathématiques appliquées et biomédecine}
    {Université Fédérale d'Uberlandia}
    {MG, Brésil}
    {3 mois}
    {Comparaison de \textbf{réseaux de neurones artificiels} en Python et en C++ pour la simulation de prothèses de membres supérieurs (bras/mains). Caractérisation de signaux électriques et application à la \textbf{réalité augmentée}.}
% \cventry{2010}
%     {Technicien informatique}
%     {Direction de l'informatique et des télécommunications du Conseil Général de la Haute Garonne}
%     {Toulouse (31)}
%     {1 mois}
%     {Tri et enrichissement de la base de données de photos du Conseil Général.}
% \cventry{Juillet 2009}
%     {Stage ouvrier}
%     {Les Fromageries Occitanes}
%     {Villefranche de Lauragais (31)}{}
%     {Travail en équipe sur une chaîne de découpe et conditionnement de fromage.}
% \cventry{2007 -- 2008}
%     {Professeur d'Aïkido}
%     {Aïkido Club des Mazades}
%     {Toulouse (31)}{}
%     {Enseignement de l'Aïkido à des enfants de 7 à 12 ans. 1h30 par semaine.}

\section{Formation scientifique}
\cventry{2008 -- 2013}
    {Diplome d'ingénieur}
    {Institut National des Sciences Appliquées (INSA)}
    {Rouen (76)}{}
    {Département Génie Mathématique.}
\cventry{Juin 2008}
    {Baccalauréat Scientifique}
    {Lycée Pierre de Fermat}
    {Toulouse (31)}{}
    {Spécialité physique chimie, mention bien.}

\section{Compétences}

\subsection{Informatique}
\cvcomputer{Programmation}{
    \textbf{Python},
    \textbf{script shell},
    C/C++,
    Fortran,
    \newline
    MPI/OpenMP,
    Java,
    UML.
    % Prolog (CHIP).
    % Pascal,
    % Perl.
}
{Outils}{
    \textbf{Git},
    % SVN,
    % Eclipse,
    % Jira,
    Jenkins,
    Gitlab CI
    \newline
    % maven,
    % Ant.
    Configure,
    Makefile.
}
\cvcomputer{Web -- SGBD}{
    HTML/CSS,
    javascript,
    PHP,
    \newline
    MySQL,
    PostgrSQL.}
{OS et logiciels}{
    GNU Linux,
    Mac OSX,
    Windows,
    \newline
    % Matlab,
    % ExtendSim,
    % OpenMP,
    MobaXterm,
    \LaTeX,
    MS Office.
}

\subsection{Mathématiques appliquées}
\cvcomputer{Recherche opérationnelle}{
    \begin{itemize}
        \item \textbf{Optimisation}
        \item \textbf{Algorithmique avancée et parallélisation}
        \item Programmation logique par contrainte
        \item Probabilités et statistiques
        % \item Processus de Markov
        % \item Optimisation combinatoire et convexe
        % \item Optimisation dans les réseaux
        \item Théorie du signal
        % \item Contrôle optimal
        % \item Automatique
    \end{itemize}
}
{Calcul scientifique}{
    \begin{itemize}
        \item Calcul différentiel
        \item Éléments finis
        \item Méthode des éléments finis
        % \item Calcul spectral
        % \item Équations aux dérivées partielles
        \item Analyse numérique
        \item Algorithmique numérique et arithmétique
    \end{itemize}
}

\subsection{Projets}
\cvline{2018 -- 2020}
    {Developpement en Java de l'application Android \href{https://gitlab.com/MisterFruits/TextLaunch}{$\hookrightarrow$ TextLaunch}, \textbf{intégration continue} avec Gitlab.}
\cvline{2012 -- 2013}
    {Conception de réseaux hiérarchiques : \textbf{programmation linéaire en nombres entiers} avec CPLEX (IBM) en Python. Application de la théorie des graphes avec Porta.}
\cvline{2012}
    {Étude de la parallélisation d'un algorithme d'\textbf{optimisation par essaim} (PSO).}
% \cvline{2011}
%     {Conception et programmation d'un jeu de go en UML et Java suivant le modèle MVC.}
% \cvline{2011}
%     {Programmation orienté objet d'un ensemble de jeux de cartes en C++. Travail en groupe à l'aide de SVN.}
\cvline{2011}
    {\textbf{Segmentation d'image}, algorithmes de partitionnement de graphe : Kernighan et Lin, Karger et Stoer-Wagner.}
% \cvline{2009}
%     {\textbf{Modélisation} de la propagation d'un virus dans une population avec Scilab.}

\section{Langues}
\cvlanguage{Anglais}{Courant, score au TOEIC 935 sur 990.}
    {Espagnol}{Compréhension écrite et orale.}
\cvlanguage{Portugais}{Courant}
    {Allemand}{Compréhension écrite et orale.}

\section{Centres d'intérêt}
% \cvline{Voyage}
%     {À but linguistique et sportif, traversée de l'atlantique en voilier, apprentissage du \textbf{portugais au Brésil} et pratique intensive de l'escalade et du surf (2018-2019)}
\cvline{Sport}{
Pratique de l'escalade% en falaise et en salle
%, du surf
 , de l'Aïkido (1er dan et enseignement à des enfants)%
 , régates de voile en équipage.%(participation à la course croisière EDHEC en 2012)
% et de la capoeira.
}{}
%\cvline{Expression corporelle}{
%Deux années de
%Jonglage, cirque et théâtre d'improvisation.
%Apprentissage et enseignement du
%Jonglage dans le club associatif de l'INSA Rouen (76).
%}{}

\end{document}
