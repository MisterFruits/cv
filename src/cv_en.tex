\documentclass[10pt,a4paper,sans]{moderncv}%sans(serif)/roman
\moderncvtheme[blue]{classic} %blue orange red gray green
\usepackage[utf8]{inputenc}
\usepackage[scale=0.85]{geometry}
%\usepackage[margin={0.6in,0.5in}]{geometry} %si limite en place
\addtolength{\textheight}{4cm}
\AtBeginDocument{\recomputelengths}


%----------------------------------------------------------------------------------
%            Commandes perso
%----------------------------------------------------------------------------------
\renewcommand*{\cvlanguage}[4]{%
  \cvdoubleitem{\textbf{#1}}{\small#2}{\textbf{#3}}{\small#4}}

% usage: \cventry{years}{degree/job title}{institution/employer}{localization}{optionnal: grade/...}{optional: comment/job description}
\renewcommand*{\cventry}[6]{%
  \cvline{#1}{%
    {\bfseries#2}%
    \ifx#3\else{, {#3}}\fi%
    \ifx#4\else{, \small#4}\fi%
    \ifx#5\else{, \small#5}\fi%
    .%
    \ifx#6\else{\newline{}\begin{minipage}[t]{\linewidth}\small#6\end{minipage}}\fi
    }}%

%----------------------------------------------------------------------------------
%            Photo et infos persos
%----------------------------------------------------------------------------------
\firstname{Victor}
\familyname{Cameo Ponz}
\title{HPC and software engineer}
\address{3 rue Tour Gayraud}{34 070 Montpellier}
\mobile{+33 6 74 08 96 90}
\email{vicameo@free.fr}
\extrainfo{31 year old | French}
%\photo[60pt]{./photo/victorID.JPG}
%\quote{Ingénieur d'étude en optimization et recherche opérationnelle, disponible immédiatement.}

%----------------------------------------------------------------------------------
%            contenu du cv
%----------------------------------------------------------------------------------
\begin{document}
\maketitle
% \vspace{-1cm} %si limite en place

\section{Experience}
\cventry{2021}
    {High-Performance Computing (HPC) engineer}
    {CINES}
    {Montpellier, France}
    {4 months}
    {
        \begin{itemize}
            \item Industrialization of software installation on a high-performance computer using the \textbf{package manager Spack and JUBE and ReFrame as QA}.
            \item Defined and tracked key performance indicators of an HPC system with \textbf{Python}.
            \item Code optimization with the European High-Level Support Team.
        \end{itemize}
    }

\cventry{2015 - 2018}
    {High-Performance Computing (HPC) engineer}
    {CINES}
    {Montpellier, France}
    {3 years}
    {
        \begin{itemize}
            \item Industrialization of software installation on a high-performance computer using the \textbf{package manager EasyBuild}.
            \item \textbf{Project leader} benchmarking the "Unified European Application Benchmark Suite (UEABS - PRACE)" with \textbf{Intel Xeon Phi, NVIDIA GPU e Maxeler FPGA} technologies -- 10 people.
%            \item Software deployment, benchmark and support no supercomputador do CINES Occigen: 90 000 cores, \#3 França, \#26 mondial.
            \item \textbf{Trainer} for software debugging and Git VCS.
        \end{itemize}
    }
\cventry{2013 -- 2015}
    {Software engineer for satellite operation products}
    {AKKA for Airbus}
    {Toulouse, France}
    {2 years}
    {
        \textbf{Development and industrialization} of a telecommunication satellite payload configuration software.
        % Generation of operational procedure and integration on control center.
         % \textbf{Java and C++}.
        \begin{itemize}
            \item \textbf{C++} optimization kernel.
            \item Operational procedure generation.
            \item Control center integration.
        \end{itemize}
    }
\cventry{2013}
    {Satellite design and optimization software trainee}
    {Airbus Defence and Space}
    {Toulouse, France}
    {6 months}
    {
        Frequency allocation optimization of a multi-beam satellite.
        % Retro-engineering and improvement of a kernel using constraint programming.
        % \textbf{Conception and developement} of a simulated annealing kernel.
        \begin{itemize}
            \item Retro-engineering and improvement of a kernel using constraint programming.
            \item \textbf{Conception and development} of a simulated annealing solver
            \item Development of visual tools
        \end{itemize}
    }
\cventry{2011}
    {Applied mathematic and biomedicine trainee}
    {Federal University of Uberlandia}
    {MG, Brazil}
    {3 months}
    {Comparison of neural networks applied to Augmented Reality for upper limb simulation.
    \begin{itemize}
        \item \textbf{Python and C++}
        \item LVQ and Backpropagation network models
    \end{itemize}
    }


\section{Extra-professional projects}
\cvline{2018 -- 2021}
    {Development with Java of an \textbf{open-source} Android launcher targeting quick access to Apps \& Contacts. CI/CD integration with Gitlab. Link to the project: \href{https://gitlab.com/MisterFruits/TextLaunch}{$\hookrightarrow$ TextLaunch}}
    % {Desenvolvimento Java do aplicativo Android \href{https://gitlab.com/MisterFruits/TextLaunch}{($\hookrightarrow$ TextLaunch)}, \textbf{integração continua} com Gitlab.}
    % }
% \cvline{2012 -- 2013}
%     {Concepção de rede hierárquicas : \textbf{programação inteira} com CPLEX Python (IBM). Aplicação da teoria dos grafos com Porta.}
% \cvline{2012}
%     {Étude de la parallélisation d'un algorithme d'\textbf{optimization par essaim} (PSO).}
% \cvline{2011}
%     {Concepção et programação de um jogo de Go com \textbf{UML e Java} seguindo o padrão \textbf{MVC}.}
% \cvline{2011}
%     {Programmation orienté objet d'un ensemble de jeux de cartes en C++. Travail en groupe à l'aide de SVN.}
% \cvline{2011}
%     {\textbf{Segmentation d'image}, algorithmes de partitionnement de graphe : Kernighan et Lin, Karger et Stoer-Wagner.}
% \cvline{2009}
%     {\textbf{Modélisation} de la propagation d'un virus dans une population avec Scilab.}


\section{Skills}
\cvcomputer{Technical}{
    \textbf{Python},
    \textbf{C/C++},
    % Java,
    MPI/OpenMP,
    % script shell,
    \newline
    \textbf{Git},
    Gitlab CI,
    Jenkins,
    \newline
    Configure/make,
    CMake,
    Ant.
    % Makefile,
    % Fortran,
    % UML.
    % Prolog (CHIP).
    % Pascal,
    % Perl.
}{Tools}{
    % Eclipse,
    Jira,
    Mattermost,
    SCRUM,
    \newline
    GNU Linux,
    \LaTeX,
    Google \& MS Office.
    % maven,
}

% \cvcomputer{Software developpement}{
%     \textbf{Python},
%     \textbf{C/C++},
%     Java,
%     % script shell,
%     \newline
%     Fortran,
%     MPI/OpenMP,
%     UML.
%     % Prolog (CHIP).
%     % Pascal,
%     % Perl.
% }
% {Tools}{
%     \textbf{Git},
%     % SVN,
%     % Eclipse,
%     % Jira,
%     Jenkins,
%     Gitlab CI
%     \newline
%     %Gradle,
%     % maven,
%     Ant,
%     Configure/make,
%     Makefile.
% }
% \cvcomputer{Web -- SGBD}{
%     HTML/CSS,
%     javascript,
%     PHP,
%     \newline
%     MySQL,
%     PostgrSQL.}
% {OS et logiciels}{
%     GNU Linux,
%     Mac OSX,
%     Windows,
%     \newline
%     % Matlab,
%     % ExtendSim,
%     \LaTeX,
%     MS Office.
% }


% \subsection{Mathé% \cvcomputer{Recherche opérationnelle}{
%     \begin{itemize}
%         \item \textbf{optimization}
%         \item \textbf{Algorithmique avancée et parallélisation}
%         \item Programmation logique par contrainte
%         \item Probabilités et statistiques
%         % \item Processus de Markov
%         % \item optimization combinatoire et convexe
%         % \item optimization dans les réseaux
%         \item Théorie du signal
%         % \item Contrôle optimal
%         % \item Automatique
%     \end{itemize}
% }
% {Calcul scientifique}{
%     \begin{itemize}
%         \item Calcul différentiel
%         \item Éléments finis
%         \item Méthode des éléments finis
%         % \item Calcul spectral
%         % \item Équations aux dérivées partielles
%         \item Analyse numérique
%         \item Algorithmique numérique et arithmétique
%     \end{itemize}
% }

\section{Languages}
\cvlanguage{English}{Professional proficiency (TOEIC 935/990).}
    {French}{Native.}
\cvlanguage{Portuguese}{Proficient.}
    {Spanish}{Intermediate.}


\section{Education}
\cventry{2008 -- 2013}
    {Master of Engineering}
    {National Institut of Applied Sciences (INSA)}
    {Rouen, France}{}
    {Specialization: computational and applied Mathematics}


\section{Hobbies}
\cvline{}{\begin{itemize}
    \item Playing Go and other board games.
    \item Organizing workshop about male contraception
    \item Outdoor activities: mountaineering, climbing.
\end{itemize}
}

% \section{Interesses}
% \cvline{Viagem}
%     {Um ano (2018-2019) com objectivo de aprender o \textbf{Português do Brasil} e a pratica intensiva da escalada e do surf.}
% % \cvline{Sport}{
% % Pratique de l'escalade% en falaise et en salle
% % , du surf
% % , de l'Aïkido (1er dan et enseignement à des enfants)%
% % , régates de voile en équipage.%(participation à la course croisière EDHEC en 2012)
% % et de la capoeira.
% % }{}
% \cvline{Expressão corporal}{
% %Deux années de
% Malabarismo, circo et teatro de improvisação.
% %Apprentissage et enseignement du
% %Jonglage dans le club associatif de l'INSA Rouen (76).
% }{}

\end{document}
