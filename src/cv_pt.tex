\documentclass[10pt,a4paper,sans]{moderncv}%sans(serif)/roman
\moderncvtheme[blue]{classic} %blue orange red gray green
\usepackage[utf8]{inputenc}
\usepackage[scale=0.85]{geometry}
%\usepackage[margin={0.6in,0.5in}]{geometry} %si limite en place
\addtolength{\textheight}{4cm}
\AtBeginDocument{\recomputelengths}


%----------------------------------------------------------------------------------
%            Commandes perso
%----------------------------------------------------------------------------------
\renewcommand*{\cvlanguage}[4]{%
  \cvdoubleitem{\textbf{#1}}{\small#2}{\textbf{#3}}{\small#4}}

% usage: \cventry{years}{degree/job title}{institution/employer}{localization}{optionnal: grade/...}{optional: comment/job description}
\renewcommand*{\cventry}[6]{%
  \cvline{#1}{%
    {\bfseries#2}%
    \ifx#3\else{, {#3}}\fi%
    \ifx#4\else{, \small#4}\fi%
    \ifx#5\else{, \small#5}\fi%
    .%
    \ifx#6\else{\newline{}\begin{minipage}[t]{\linewidth}\small#6\end{minipage}}\fi
    }}%

%----------------------------------------------------------------------------------
%            Photo et infos persos
%----------------------------------------------------------------------------------
\firstname{Victor}
\familyname{Cameo Ponz}
\title{Engenheiro em matemática aplicada e desenvolvedor}
%\address{13 rue des Peupliers}{34830 Clapiers}
\mobile{+33 06 74 08 96 90 (WhatsApp)}
%\mobile{+55 21 99 365 2969 (Brésil)}
\email{vicameo@free.fr}
\extrainfo{29 anos | Frances}
%\photo[60pt]{./photo/victorID.JPG}
%\quote{Ingénieur d'étude en optimisation et recherche opérationnelle, disponible immédiatement.}

%----------------------------------------------------------------------------------
%            contenu du cv
%----------------------------------------------------------------------------------
\begin{document}
\maketitle
\vspace{-1cm} %si limite en place

\section{Atuacão profissional}
\cventry{2015 - 2018}
    {Engenheiro em computação de alto desempenho (HPC)}
    {CINES}
    {Montpellier, França}
    {3 anos}
    {
        \begin{itemize}
            \item \textbf{Lider do projeto} de evaluacão das performencias da "Unified European Application Benchmark Suite (UEABS - PRACE)" nas technologias \textbf{Intel Xeon Phi, NVIDIA GPU e Maxeler FPGA} -- 10 pessoas.
            \item Instalação de software de pesquisa, manutenção, benchmark e suporte no supercomputador do CINES Occigen: 90 000 cores, #3 França, #26 mondial.
            \item Formador em varios assuntos: Computação paralela e debugging (sessões de 4 dias), Workshop Intel Xeon Phi (2 dias), \textbf{Git} (1 dia), iniciação ao HPC (meio dia).
            \item Adaptação e optimização de \textbf{applicativos industrias} para o HPC.
    }
\cventry{2013 -- 2015}
    {Engenheiro software e produtos de operacão satélite}
    {AKKA para Airbus}
    {Toulouse, França}
    {2 anos}
    {
        Desenvolvimento e produção de software para reconfigurar a carga útil de um \textbf{satélite de telecomunicações}.
        \begin{itemize}
            \item Nucleo de optimização em \textbf{C++}, operado com o aplicativo \textbf{Java via JNI}.
            \item Geração de procedimentos operacionais com templates e integração no centro de controle
        \end{itemize}
    }
\cventry{2013}
    {Estagio engenheiro, produto de otimização de design satélite}
    {EADS Astrium}
    {Toulouse, França}
    {6 meses}
    {
        Otimização da \textbf{alocação de frequência} de um satélite de \textbf{feixe múltiplo}.
        \begin{itemize}
            \item Retro-engenharia e aprimoramento de um kernel em \textbf{programação por restrições}.
            \item \textbf{Concepção e realização} de um solucionador de recozimento simulado.
            \item Desenvolvimento de ferramentas de visualização com JFreeChart e World Wind Java.
        \end{itemize}
    }
\cventry{2011}
    {Estagio matemática aplicada a biomedicine}
    {Universidade Federal de Uberlandia}
    {MG, Brasil}
    {3 meses}
    {Comparação de \textbf{rede neurais artificiais} em Python e C++ na simulação de proteses de mão e braso. Clasificação de sinais eletromiográficos e aplicação a \textbf{realidade virtual}.}


\section{Formation scientifique}
\cventry{2008 -- 2013}
    {Mestrado em Engenharia Matemática}
    {Instituto Nacional de Ciências Aplicadas (INSA Rouen)}
    {Rouen, França}{}
    {Especialização em optimização e Algoritmos avançados e paralelismo}


\section{Habilidades informaticas}
\cvcomputer{Programação}{
    Python,
    C/C++,
    Java,
    % script shell,
    \newline
    Fortran,
    MPI/OpenMP,
    UML.
    % Prolog (CHIP).
    % Pascal,
    % Perl.
}
{Ferramentas}{
    \textbf{Git},
    % SVN,
    % Eclipse,
    % Jira,
    Jenkins,
    Gitlab CI
    \newline
    Gradle
    % maven,
    % Ant.
    Configure,
    Makefile.
}
\cvcomputer{Web -- SGBD}{
    HTML/CSS,
    javascript,
    PHP,
    \newline
    MySQL,
    PostgrSQL.}
{OS et logiciels}{
    GNU Linux,
    Mac OSX,
    Windows,
    \newline
    % Matlab,
    % ExtendSim,
    \LaTeX,
    MS Office.
}

% \subsection{Mathé% \cvcomputer{Recherche opérationnelle}{
%     \begin{itemize}
%         \item \textbf{Optimisation}
%         \item \textbf{Algorithmique avancée et parallélisation}
%         \item Programmation logique par contrainte
%         \item Probabilités et statistiques
%         % \item Processus de Markov
%         % \item Optimisation combinatoire et convexe
%         % \item Optimisation dans les réseaux
%         \item Théorie du signal
%         % \item Contrôle optimal
%         % \item Automatique
%     \end{itemize}
% }
% {Calcul scientifique}{
%     \begin{itemize}
%         \item Calcul différentiel
%         \item Éléments finis
%         \item Méthode des éléments finis
%         % \item Calcul spectral
%         % \item Équations aux dérivées partielles
%         \item Analyse numérique
%         \item Algorithmique numérique et arithmétique
%     \end{itemize}
% }

\section{Projetos extra-profissionais}
\cvline{2018 -- 2019}
    {Desenvolvimento Java do aplicativo Android \href{https://gitlab.com/MisterFruits/TextLaunch}{($\hookrightarrow$ TextLaunch)}, \textbf{integração continua} com Gitlab.}
% \cvline{2012 -- 2013}
%     {Concepção de rede hierárquicas : \textbf{programação inteira} com CPLEX Python (IBM). Aplicação da teoria dos grafos com Porta.}
% \cvline{2012}
%     {Étude de la parallélisation d'un algorithme d'\textbf{optimisation par essaim} (PSO).}
\cvline{2011}
    {Concepção et programação de um jogo de Go com UML e Java seguindo o padrão MVC.}
% \cvline{2011}
%     {Programmation orienté objet d'un ensemble de jeux de cartes en C++. Travail en groupe à l'aide de SVN.}
% \cvline{2011}
%     {\textbf{Segmentation d'image}, algorithmes de partitionnement de graphe : Kernighan et Lin, Karger et Stoer-Wagner.}
% \cvline{2009}
%     {\textbf{Modélisation} de la propagation d'un virus dans une population avec Scilab.}

\section{Linguas}
\cvlanguage{Português do Brasil}{Proficiência profissional plena.}
    {Francês}{Falante nativo}
\cvlanguage{Inglês}{Proficiência profissional plena (TOEIC 935/990).}
    {Espanhol}{Compreensão escrita e oral.}

\section{Interesses}
\cvline{Viagem}
    {Um ano (2018-2019) com objectivo de aprender o \textbf{Português do Brasil} e a pratica intensiva da escalada e do surf}
% \cvline{Sport}{
% Pratique de l'escalade% en falaise et en salle
% , du surf
% , de l'Aïkido (1er dan et enseignement à des enfants)%
% , régates de voile en équipage.%(participation à la course croisière EDHEC en 2012)
% et de la capoeira.
% }{}
\cvline{Expressão corporal}{
%Deux années de
Malabarismo, circo et teatro de improvisação.
%Apprentissage et enseignement du
%Jonglage dans le club associatif de l'INSA Rouen (76).
}{}

\end{document}
