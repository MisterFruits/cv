%%Stage ingénieur de 6 mois à partir d'avril 2013.
%%
%%Je cherche un stage ayant un fort caractère mathématiques. Optimisation, calcul numérique,  traitement du signal, algorithmique, calcul de complexité... C'est la recherche opérationnelle qui m'intéresse.
%%
%%Ma formation  donne accès à toutes les base de physique : mécanique, thermodynamique, chimie, science de l'ingénieur. ma spécialisation me donne Un plus en informatique en plus des maths appliquéesé
%%
%%Je cherche à mettre en pratique mes acquis particulièrement dans le domaine de ...
 

\documentclass[12pt]{lettre}
 
\usepackage[utf8]{inputenc}
\usepackage[T1]{fontenc}
%\usepackage{lmodern}
\usepackage[frenchb]{babel}


\usepackage{eurosym}
%%%%%%%%%%%%%%%%%%%%%%%%%%%%%%%
%%% LE FORMAT DE LA PAGE
%%%%%%%%%%%%%%%%%%%%%%%%%%%%%%%
%\geometry
%   {%
%   paper=a4paper,%
%   margin=3cm,%
%   hoffset=0cm,%
   %headsep=0.5cm,%
%   }%
%%%%%%%%%%%%%%%%%%%%%%%%%%%%%%%
%%% EN-TÊTES 
%%%%%%%%%%%%%%%%%%%%%%%%%%%%%%%
%\pagestyle{empty}
%%%%%%%%%%%%%%%%%%%%%%%%%%%%%%%
%%% OPTIONS DES TABLEAUX
%%%%%%%%%%%%%%%%%%%%%%%%%%%%%%%
%\setlength{\tabcolsep}{1pt}
%%%%%%%%%%%%%%%%%%%%%%%%%%%%%%%
%%% OPTIONS DIVERSES : math, soulignement
%%%%%%%%%%%%%%%%%%%%%%%%%%%%%%%
\parindent=6ex

%\usepackage[margin={0.6in,0.5in}]{geometry}
\addtolength{\textheight}{4cm} 
%\AtBeginDocument{\recomputelengths} 

 
\begin{document}
 
\begin{letter}{A l'attention du service de recrutement}
\name{Victor Cameo Ponz}
\address{Victor Cameo Ponz\\47 rue Guynemer\\31200 Toulouse}
\email{victor.cameo\_ponz@insa-rouen.fr}
\telephone{+33 (0)6.74.08.96.90}
\lieu{Toulouse}
\nofax

\conc{Proposition de collaboration avec Astrium}
\opening{Madame, Monsieur}
%\setlength{\parindent}{\openingindent}%corps

Fraichement sorti de l'Institut National des Sciences Appliquées (INSA) de Rouen, j'aimerai avoir une expérience dans le domaine de l'enseignement.

Le stage que j'ai effectué à Astrium au cours de ces six derniers mois a conforté ma volonté de travailler dans le domaine des mathématiques appliquées et de l'optimisation. Son exécution m'a permis de découvrir les problématiques complexes de l'industrie spatiale qui m'étaient inconnues. J'ai eu la chance d'intégrer l'équipe de management et d'optimisation de la charge utile des satellites de télécommunication. J'y ai développé en autonomie un algorithme d'optimisation ainsi que des outils de visualisation et d'analyse de résultats.

La formation que j'ai suivie à l'INSA m'a donné de fortes compétences en modélisation mathématique, abstraction et de généralisation. Je suis dédié à travaillé dans des domaines transverses grâce à ma capacité à transposer une méthode d'une discipline à une autre. Les projets et stages que j'ai effectué au cours de ma formation en sont l'illustration parfaite. De plus, les compétences en informatique et gestion de projet que j'ai développé me permettent d'être à l'aise et efficace dans tous types d'environnements techniques.% Elle sont un véritable atout dans les applications que j'ai pu de faire de l'enseignement qu'on m'a donné.

%Je cherche à varier mon expérience professionnelle et intégrer une entreprise à taille humaine comme Ippon Inovation semble être une bonne opportunité pour y investir mes connaissances et mon dynamisme.
%
Ma première expérience dans le monde du spatiale a été une expérience enrichissante et motivant et l'opportunité d'intégrer une entreprise  telle que Thales me permettrait de m'y investir complètement.
%
Je me tiens à votre disposition pour toutes informations complémentaire ou un entretient.
\closing{Je vous prie d'agréer, Madame, Monsieur, l'expression de mes sentiments distingués.}
%\encl{Curiculum Vitae}
\end{letter}
 
\end{document}