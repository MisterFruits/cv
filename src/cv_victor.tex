\documentclass[10pt,a4paper,sans]{moderncv}%sans(serif)/roman
\moderncvtheme[blue]{classic} %blue orange red gray green
\usepackage[utf8]{inputenc}
\usepackage[scale=0.85]{geometry}
%\usepackage[margin={0.6in,0.5in}]{geometry} %si limite en place
\addtolength{\textheight}{4cm}
\AtBeginDocument{\recomputelengths}


%----------------------------------------------------------------------------------
%            Commandes perso
%----------------------------------------------------------------------------------
\renewcommand*{\cvlanguage}[4]{%
  \cvdoubleitem{\textbf{#1}}{\small#2}{\textbf{#3}}{\small#4}}

%----------------------------------------------------------------------------------
%            Photo et infos persos
%----------------------------------------------------------------------------------
\firstname{Victor}
\familyname{Cameo Ponz}
%\title{Élève ingénieur à l'INSA \newline \mbox{Génie Mathématique}}
\title{Ingénieur INSA, \mbox{Génie Mathématique}}
\address{47 rue Guynemer}{31200 Toulouse}
\mobile{+33 (0)6 74 08 96 90}
\email{victor.cameo\_ponz@insa-rouen.fr}
\extrainfo{Titulaire du permis B, 23 ans}
%\photo[60pt]{./photo/victorID.JPG}
%\quote{Ingénieur d'étude en optimisation et recherche opérationnelle, disponible immédiatement.}

%----------------------------------------------------------------------------------
%            contenu du cv
%----------------------------------------------------------------------------------
\begin{document}
\maketitle
\vspace{-1cm} %si limite en place

\section{Formation scientifique}
\cventry{2008 -- 2013}
    {Diplome d'ingénieur}
    {Institut National des Sciences Appliquées (INSA)}
    {Rouen (76)}{}
    {Département Génie Mathématique.}
\cventry{Juin 2008}
    {Baccalauréat Scientifique}
    {Lycée Pierre de Fermat}
    {Toulouse (31)}{}
    {Spécialité physique chimie, mention bien.}

\section{Expérience professionnelle}
\cventry{2013}
    {EADS Astrium Central Engineering}{}
    {Toulouse (31)}{}
    {Optimisation de \textbf{l'allocation fréquentielle} d'un \textbf{satellite multi-beam}. Retro-engineering et amélioration d'un solveur CHOCO. \textbf{Conception et réalisation d'un solveur} recuit simulé. Interface graphique : JFreeChart, World Wind Java et Excel. \textit{Utilisation  de GIT, Java, Eclipse, TPTP et VBA.} Stage ingénieur, 6 mois.}
\cventry{2011 -- 2012}
    {Professeur particulier}{}
    {Rouen (76)}{}
    {Cours particuliers de mathématiques.}
\cventry{2011}
    {Universidade Federal de Uberlandia}{}
    {Uberlandia - MG, Brésil}{}
    {Comparaison de \textbf{réseaux de neurones artificiels} et application à la \textbf{réalité augmentée} pour la simulation de prothèses de membres supérieurs (bras/mains). \textit{C++ et Python}. Stage technicien, 3 mois.}
\cventry{2010}
    {Technicien informatique}
    {Direction de l'informatique et des télécommunications du Conseil Général de la Haute Garonne}
    {Toulouse (31)}{}
    {Tri et enrichissement de la base de données du Conseil Général. 1 mois.}
% \cventry{Juillet 2009}
%     {Stage ouvrier}
%     {Les Fromageries Occitanes}
%     {Villefranche de Lauragais (31)}{}
%     {Travail en équipe sur une chaîne de découpe et conditionnement de fromage.}
% \cventry{2007 -- 2008}
%     {Professeur d'Aïkido}
%     {Aïkido Club des Mazades}
%     {Toulouse (31)}{}
%     {Enseignement de l'Aïkido à des enfants de 7 à 12 ans. 1h30 par semaine.}

\section{Compétences}
\subsection{Mathématiques appliquées}
\cvline{Recherche opérationnelle}{
    % Probabilités et statistiques,
    %processus de Markov,
    Optimisation,
    %optimisation combinatoire et convexe,
    %optimisation dans les réseaux,
    algorithmique avancée et parallélisation,
    %programmation logique par contrainte,
    théorie du signal,
    %contrôle optimal,
    automatique.
}
\cvline{Calcul scientifique}{
    Calcul différentiel,
    éléments finis,
    %méthode des éléments finis,
    calcul spectral,
    %équations aux dérivées partielles,
    analyse numérique.
    %algorithmique numérique et arithmétique.
}

\subsection{Projets}
\cvline{2012 -- 2013}
    {Conception de réseaux hiérarchiques, \textbf{programmation linéaire en nombres entiers}, optimisation dans les graphes. Utilisation de CPLEX (IBM), Porta, Python}
\cvline{2012}
    {Étude de la parallélisation d'un algorithme d'\textbf{optimisation par essaim} (PSO).}
%\textbf{2011} : Conception et programmation d'un jeu de go en Java suivant le modèle MVC. \textit{Eclipse, Topcased}.\newline
\cvline{2011}
    {Programmation orienté objet d'un ensemble de jeux de cartes. C++, SVN.}
\cvline{2011}
    {\textbf{Segmentation d'image}, algorithmes de partitionnement de graphe. Kernighan et Lin, Karger et Stoer-Wagner.}
\cvline{2009}
    {\textbf{Modélisation} de la propagation d'un virus dans une population. Scilab.}

\subsection{Informatique}
\cvcomputer{Programmation}{
    Java,
    C/C++,
    UML,
    Python,
    \newline
    Fortran,
    script shell,
    %Prolog (CHIP),
    %Pascal,
    Perl.
}
{OS et logiciels}{
    GNU Linux,
    Mac OSX,
    Windows,
    \newline
    Matlab,
    ExtendSim,
    %OpenMP,
    %\LaTeX,
    MS Office.
}
\cvcomputer{Web}{HTML/CSS, PHP, javascript.}
    {SGBD}{MySQL, PostgrSQL.}
\subsection{Langues}
\cvlanguage{Anglais}{Courant, score au TOEIC 935 sur 990.}
    {Espagnol}{Compréhension écrite et orale.}
\cvlanguage{Portugais}{Conversation et compréhension écrite.}
    {Allemand}{Compréhension écrite et orale.}

\section{Centres d'intérêt}
\cvline{Sport}{
Pratique de l'escalade% en falaise et en salle
, de l'Aïkido (1er dan et enseignement à des enfants)%
, régates de voile en équipage.%(participation à la course croisière EDHEC en 2012)
}{}
\cvline{Vie associative}{
%Deux années de
Jonglage et théâtre étude à l'INSA. Participation dans une pièce en anglais, \textit{Still Kidin' (Univ. de Rouen)}.\newline
%Apprentissage et enseignement du
%Jonglage dans le club associatif de l'INSA Rouen (76).
}{}

\end{document}
