\documentclass[10pt,a4paper,sans]{moderncv}%sans(serif)/roman
\moderncvtheme[blue]{classic} %blue orange red gray green
\usepackage[utf8]{inputenc}
\usepackage[scale=0.85]{geometry}
%\usepackage[margin={0.6in,0.5in}]{geometry} %si limite en place
\addtolength{\textheight}{4cm}
\AtBeginDocument{\recomputelengths}


%----------------------------------------------------------------------------------
%            Commandes perso
%----------------------------------------------------------------------------------
\renewcommand*{\cvlanguage}[4]{%
  \cvdoubleitem{\textbf{#1}}{\small#2}{\textbf{#3}}{\small#4}}

% usage: \cventry{years}{degree/job title}{institution/employer}{localization}{optionnal: grade/...}{optional: comment/job description}
\renewcommand*{\cventry}[6]{%
  \cvline{#1}{%
    {\bfseries#2}%
    \ifx#3\else{, {#3}}\fi%
    \ifx#4\else{, \small#4}\fi%
    \ifx#5\else{, \small#5}\fi%
    .%
    \ifx#6\else{\newline{}\begin{minipage}[t]{\linewidth}\small#6\end{minipage}}\fi
    }}%

%----------------------------------------------------------------------------------
%            Photo et infos persos
%----------------------------------------------------------------------------------
\firstname{Victor}
\familyname{Cameo Ponz}
%\title{Élève ingénieur à l'INSA \newline \mbox{Génie Mathématique}}
\title{Ingénieur INSA, mathématiques appliquées}
\address{47 rue Guynemer}{31200 Toulouse}
% \mobile{06 74 08 96 90}
\mobile{+33 (0)6 74 08 96 90}
\email{victor.cameo\_ponz@insa-rouen.fr}
\extrainfo{Titulaire du permis B, 23 ans}
%\photo[60pt]{./photo/victorID.JPG}
%\quote{Ingénieur d'étude en optimisation et recherche opérationnelle, disponible immédiatement.}

%----------------------------------------------------------------------------------
%            contenu du cv
%----------------------------------------------------------------------------------
\begin{document}
\maketitle
\vspace{-1cm} %si limite en place

\section{Expérience professionnelle}
\cventry{Depuis 2013}
    {Développeur Java et C++}
    {AKKA pour Airbus Denfence and Space}
    {Toulouse (31)}
    {1 an}
    {
        Développement et production d'un logiciel de \textbf{reconfiguration de la charge utile} d'un satellite.
        \begin{itemize}
            \item Génération de procédures opérationnelles
            \item Intégration sur centre de contrôle
            \item Noyau d'optimisation en C++
        \end{itemize}
    }
\cventry{2013}
    {Stage en optimisation}
    {EADS Astrium Central Engineering}
    {Toulouse (31)}
    {6 mois}
    {
        Optimisation de \textbf{l'allocation fréquentielle} d'un \textbf{satellite multi-beam}.
        \begin{itemize}
            \item Retro-engineering et amélioration d'un noyau codé en programmation par contraintes.
            \item \textbf{Conception et réalisation d'un solveur} recuit simulé.
            \item Développement d'outils de visualisation avec JFreeChart et World Wind Java.
        \end{itemize}
    }
% \cventry{2011 -- 2012}
%     {Professeur particulier}{}
%     {Rouen (76)}{}
%     {Cours particuliers de mathématiques.}
\cventry{2011}
    {Stage en mathématiques appliquées}
    {Universidade Federal de Uberlandia}
    {MG, Brésil}
    {3 mois}
    {Comparaison de \textbf{réseaux de neurones artificiels} en Python et en C++. Application à la \textbf{réalité augmentée} pour la simulation de prothèses de membres supérieurs (bras/mains)}
\cventry{2010}
    {Technicien informatique}
    {Direction de l'informatique et des télécommunications du Conseil Général de la Haute Garonne}
    {Toulouse (31)}
    {1 mois}
    {Tri et enrichissement de la base de données de photos du Conseil Général.}
% \cventry{Juillet 2009}
%     {Stage ouvrier}
%     {Les Fromageries Occitanes}
%     {Villefranche de Lauragais (31)}{}
%     {Travail en équipe sur une chaîne de découpe et conditionnement de fromage.}
% \cventry{2007 -- 2008}
%     {Professeur d'Aïkido}
%     {Aïkido Club des Mazades}
%     {Toulouse (31)}{}
%     {Enseignement de l'Aïkido à des enfants de 7 à 12 ans. 1h30 par semaine.}

\section{Formation scientifique}
\cventry{2008 -- 2013}
    {Diplome d'ingénieur}
    {Institut National des Sciences Appliquées (INSA)}
    {Rouen (76)}{}
    {Département Génie Mathématique.}
\cventry{Juin 2008}
    {Baccalauréat Scientifique}
    {Lycée Pierre de Fermat}
    {Toulouse (31)}{}
    {Spécialité physique chimie, mention bien.}

\section{Compétences}
\subsection{Mathématiques appliquées}
\cvcomputer{Recherche opérationnelle}{
    \begin{itemize}
        % \item Probabilités et statistiques
        % \item Processus de Markov
        \item Optimisation
        % \item Optimisation combinatoire et convexe
        % \item Optimisation dans les réseaux
        \item Algorithmique avancée et parallélisation
        % \item Programmation logique par contrainte
        \item Théorie du signal
        % \item Contrôle optimal
        \item Automatique
    \end{itemize}
}
{Calcul scientifique}{
    \begin{itemize}
        \item Calcul différentiel
        \item Éléments finis
        % \item Méthode des éléments finis
        \item Calcul spectral
        % \item Équations aux dérivées partielles
        \item Analyse numérique
        \item Algorithmique numérique et arithmétique
    \end{itemize}
}

\subsection{Projets}
\cvline{2012 -- 2013}
    {Conception de réseaux hiérarchiques : \textbf{programmation linéaire en nombres entiers} avec CPLEX (IBM) en Python. Application de la théorie des graphes avec Porta.}
\cvline{2012}
    {Étude de la parallélisation d'un algorithme d'\textbf{optimisation par essaim} (PSO).}
% \cvline{2011}
%     {Conception et programmation d'un jeu de go en UML et Java suivant le modèle MVC.}
\cvline{2011}
    {Programmation orienté objet d'un ensemble de jeux de cartes en C++. Travail en groupe à l'aide de SVN.}
\cvline{2011}
    {\textbf{Segmentation d'image}, algorithmes de partitionnement de graphe : Kernighan et Lin, Karger et Stoer-Wagner.}
% \cvline{2009}
%     {\textbf{Modélisation} de la propagation d'un virus dans une population avec Scilab.}

\subsection{Informatique}
\cvcomputer{Programmation}{
    Java,
    Python,
    C/C++,
    UML,
    \newline
    Fortran,
    script shell,
    Prolog (CHIP).
    %Pascal,
    % Perl.
}
{Outils}{
    Git,
    % SVN,
    Eclipse,
    Jira,
    \newline
    maven,
    Jenkins
    Ant.
    % Makefile.
}
\cvcomputer{Web -- SGBD}{
    HTML/CSS,
    PHP,
    javascript,
    \newline
    MySQL,
    PostgrSQL.}
{OS et logiciels}{
    GNU Linux,
    Mac OSX,
    Windows,
    \newline
    Matlab,
    ExtendSim,
    %OpenMP,
    %\LaTeX,
    MS Office.
}
\subsection{Langues}
\cvlanguage{Anglais}{Courant, score au TOEIC 935 sur 990.}
    {Espagnol}{Compréhension écrite et orale.}
\cvlanguage{Portugais}{Conversation et compréhension écrite.}
    {Allemand}{Compréhension écrite et orale.}

\section{Centres d'intérêt}
\cvline{Sport}{
Pratique de l'escalade% en falaise et en salle
, de l'Aïkido (1er dan et enseignement à des enfants)%
, régates de voile en équipage.%(participation à la course croisière EDHEC en 2012)
}{}
\cvline{Vie associative}{
%Deux années de
Jonglage et théâtre étude à l'INSA. Participation dans une pièce en anglais, \textit{Still Kidin' (Univ. de Rouen)}.\newline
%Apprentissage et enseignement du
%Jonglage dans le club associatif de l'INSA Rouen (76).
}{}

\end{document}
