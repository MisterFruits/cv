%%Stage ingénieur de 6 mois à partir d'avril 2013.
%%
%%Je cherche un stage ayant un fort caractère mathématiques. Optimisation, calcul numérique,  traitement du signal, algorithmique, calcul de complexité... C'est la recherche opérationnelle qui m'intéresse.
%%
%%Ma formation  donne accès à toutes les base de physique : mécanique, thermodynamique, chimie, science de l'ingénieur. ma spécialisation me donne Un plus en informatique en plus des maths appliquéesé
%%
%%Je cherche à mettre en pratique mes acquis particulièrement dans le domaine de ...


\documentclass[12pt]{lettre}

\usepackage[utf8]{inputenc}
\usepackage[T1]{fontenc}
%\usepackage{lmodern}
\usepackage[frenchb]{babel}


\usepackage{eurosym}
%%%%%%%%%%%%%%%%%%%%%%%%%%%%%%%
%%% LE FORMAT DE LA PAGE
%%%%%%%%%%%%%%%%%%%%%%%%%%%%%%%
%\geometry
%   {%
%   paper=a4paper,%
%   margin=3cm,%
%   hoffset=0cm,%
   %headsep=0.5cm,%
%   }%
%%%%%%%%%%%%%%%%%%%%%%%%%%%%%%%
%%% EN-TÊTES
%%%%%%%%%%%%%%%%%%%%%%%%%%%%%%%
%\pagestyle{empty}
%%%%%%%%%%%%%%%%%%%%%%%%%%%%%%%
%%% OPTIONS DES TABLEAUX
%%%%%%%%%%%%%%%%%%%%%%%%%%%%%%%
%\setlength{\tabcolsep}{1pt}
%%%%%%%%%%%%%%%%%%%%%%%%%%%%%%%
%%% OPTIONS DIVERSES : math, soulignement
%%%%%%%%%%%%%%%%%%%%%%%%%%%%%%%
\parindent=6ex

%\usepackage[margin={0.6in,0.5in}]{geometry}
\addtolength{\textheight}{4cm}
%\AtBeginDocument{\recomputelengths}


\begin{document}

\begin{letter}{}
\name{Victor Cameo Ponz}
\address{Victor Cameo Ponz\\47 rue Guynemer\\31200 Toulouse}
\email{victor.cameo\_ponz@insa-rouen.fr}
\telephone{+33 (0)6.74.08.96.90}
\lieu{Toulouse}
\nofax

\conc{}
\opening{Bonjour Natalie,}
%\setlength{\parindent}{\openingindent}%corps
Je suis actuellement en stage sur les activités Bebop avec Bertrand Cabon et Fawzi Bessaih
(optimisation d’allocation fréquentielle pour satellite multibeams).
Suite aux nombreuses réunions que nous avons eues avec votre équipe et en
discutant récemment avec Vincent Tugend,
il apparait que de nombreux sujets relatifs à l’optimisation de design des systèmes
multibeam restent à approfondir : calcul de la disposition d’un layout avec taille de beams variable,
prise en compte des allocations de porteuses dans les dimensionnements système,
optimisation de demande de capacités etc.
Ces sujets sont au cœur de l’activité de proposition d’Astrium,
et il me semble que ces axes sont aujourd’hui stratégiques pour la BU Télécom.

Le travail que j’effectue depuis bientôt 5 mois à ACE8 avec votre équipe a été une expérience très enrichissante. Il m’a permis de découvrir un environnement multi-disciplinaire très riche, de rencontrer des équipes compétentes et motivées, et de développer des premières solutions dont les résultats sont très encourageants.

Passionné par l’algorithmie et l’optimisation, j’ai récemment demandé à Bertrand s’il y avait des opportunités de réaliser un doctorat dans ce domaine à Astrium. Continuer sur le sujet que j’ai pu débuter pendant mon stage serait bien sûr extrêmement motivant et me permettrait de continuer à développer des méthodes à forte valeur ajoutée pour la compétitivité d’Astrium. Bertrand a déjà suivi plusieurs doctorats sur le thème de l’optimisation, dont celle de Fawzi qu’il avait initiée avec Sylvain Leconte. Il connait bien le fonctionnement de ce genre de projet et son rôle d’expert au sein de l’entreprise est pour moi un gage de sérieux et de réussite industrielle pour un tel projet.

Me serait-il possible de vous rencontrer pour identifier les éventuelles opportunités de doctorat sur les activités d’optimisation de design multibeams. Je me tiens bien sûr à votre entière disposition.

\closing{}
%\encl{Curiculum Vitae}
\end{letter}

\end{document}
